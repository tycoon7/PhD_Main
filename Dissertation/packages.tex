\usepackage[square,sort&compress,numbers]{natbib} % Provides formatting for
                                                  % citations
\usepackage{textcomp} % Provides math symbols that can be used in text mode
\usepackage{amssymb}  % Provides additional AMS math symbols.  Note that
                      % amsmath is loaded as part of the afit-etd class
\usepackage{bm}       % Provides bold-faced math symbols
\usepackage{booktabs} % Provides improved table formatting
\usepackage{dcolumn}  % Provides table columns aligned at decimal points
\usepackage{multirow} % Provides table elements spanning multiple rows
\usepackage{graphicx} % Standard package to incorporate graphics
\usepackage[printonlyused]{acronym} % Provides a method for incorporating
                                    % acronyms and building an acronym list
\usepackage{subfigure} % Create subfigures
%\usepackage{subfig}
\usepackage{calc}      % allows simple and easy calculations
\usepackage{wrapfig}   % wrapped text around figures - perhaps not appropriate
                       % in a thesis, but useful in general
\usepackage[numbered]{mcode} % easily integrates matlab code
\usepackage{float} %H places the figure or tables in that exact location
\usepackage{amsthm}
\newtheorem*{definition}{Definition}
\usepackage{pdfpages}



%\usepackage{ifpdf} % \ifpdf ... \else ... fi structure
%\ifpdf 
%  \usepackage[pdftex, bookmarks, breaklinks,
%              plainpages=false,% Make page anchors using the formatted form of 
%                               % the page number. With this option, hyperref 
%                               % writes different anchors for pages 'ii' and
%                               % '2'. (If the option is set 'true' - the 
%                               % default - hyperref writes page anchors as the
%                               % arabic form of the absolute page number, 
%                               % rather than the formatted form.) [UK TeXfaq]
%              pdfpagelabels,   % Set PDF page labels; i.e., write the
%                               % value of \thepage to the PDF file so that
%                               % Acrobat Reader can display the page number as
%                               % (say) 'ii (4 of 40)' rather than simply '4 of
%                               % 40'. [UK TeXfaq]
%              colorlinks,      % use color instead of boxes for links
%              linkcolor=black, % color internal links black
%              urlcolor=black,  % color url's black
%              citecolor=black, % color links to references black
%              ]{hyperref} 
%\else  
%  \usepackage[hypertex]{hyperref}
%\fi
