Chromotomographic imaging (CTI) offers advantages in remote sensing by resolving intensity distribution spatially, spectrally, and temporally. The \acf{CTEx} at the \acf{AFIT} explores the application of CTI as a space-based observer. Previous work in instrument development has revealed many of the intricacies of component fabrication and how they impact the resolving of image data. The proposed \ac{CTEx} instrument has as its chromatic dispersion element a \ac{DVP} that is made to rotate in order to achieve multiple projection angles. The inability to reconstruct a fast-transient scene has been largely attributed to the fabrication and alignment imperfections in the \ac{DVP} and so inspire improvement of the techniques for realizing a spinning \ac{DVP} optical element. The research herein presents an investigation into precise characterization of a \ac{DVP} and a proposed mechanical design of the \ac{DVP} hardware. The findings of this investigation provide the tools to specify fabrication tolerances for the \ac{DVP} and advance the research effort.

