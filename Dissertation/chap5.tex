\chapter{Conclusions and Recommendations}
\label{ch:conclusion}

The \acf{AFIT} \acf{CTEx} is an effort to demonstrate \acf{CTI} technology on a space-based platform. The proposed \ac{CTEx} instrument has as its chromatic dispersion element a \acf{DVP} that is made to rotate in order to achieve multiple projection angles. The primary advantage of \ac{CTI} over alternative imaging technologies is its capability for spectroscopy over a wide and diverse \acf{FOV} with high temporal resolution. Successful demonstration of \ac{CTI} using prototype \ac{CTEx} instruments has extracted spatially- and spectrally-resolved image data in the visible spectrum.  The prototype \ac{CTEx} instruments have yet been unable to demonstrate the \ac{CTI} capability for temporal resolution, though attempts have been made. It was determined by analysis that the main obstacle prohibiting successful demonstration of temporal capabilities is the degradation in optical performance as a result of mechanical misalignments. It was confirmed by analysis in Section~\ref{sec:sensitiveHawks}, that small mechanical misalignments of the \ac{DVP} assembly contribute debilitating error to the reconstructed images. Past efforts in \ac{DVP} characterization implementing modified mechanical measurement techniques have proven incapable of \ac{DVP} characterization to a precision that enables the reconstruction algorithm to reliably reconstruct spectrally-continuous image data. Demonstrating reconstruction of a spectrally-continuous scene is the next step in achieving the desired temporal resolution.

% (I believe there has been additional guess-and-check in addition to the modified dispersion curves. Also, I don't see that a spectrally-continuous static scene was imaged. It is possible that the reconstruction of the spectrally-discrete scenes (Hg pen lamp) were only possible because the artifacts of nearby emission lines were not significant. In which case, I would suggest that the uncertainty in the prism dispersion strongly contributes to the difficulty with the ATK test reconstruction. I think that the cone of missing information becomes just a few streaks of missing information for Hg spectral emission.)

The research reported by this thesis was motivated by the need to build a better \ac{DVP} and characterize the as-built configuration of an assembled \ac{DVP}. Given this objective, the natural inclination was to begin testing fabrication methods and establish techniques to produce precise alignment. It was soon realized that to establish tolerances for alignment fixtures and quantify the performance of new fabrication techniques, a method for precise characterization of a prism assembly had to be established. As a result of the investigation, methods for \ac{DVP} characterization were developed.

%%%%%%%%%%%%%%%%%%%%%%%%%%%%%%%%%%%%%%%%%%%%%%%%%%%%%%%%%%%%%%%%%%%%%%%%%%%%%%%%%% 
\section{DVP Characterization Investigation Conclusions}

The \ac{DVP} characterization investigation incorporated analysis and testing to propose and validate measurement methods and data processing algorithms. The specific investigations are reviewed in this section and reveal a natural progression of conclusions.

\subsection{Misalignment Uniqueness Investigation Results}
In Section~\ref{sec:misalignmentUniqueness} the uniqueness of the output ray angles resulting from prism misalignments was investigated. Zemax was used to simulate the effects of prism misalignments in a two-prism system and the output ray angles were represented by the Zernike coefficients extracted from the simulation. It was shown in Section~\ref{sec:zernikes} that a unique combination of Zernike coefficients indicated a unique output ray angle. By inspection of Figures~\ref{fig:XRotZernikes} through~\ref{fig:NormalRotZernikes} it is evident that the same pair of coefficient values does not occur twice within the range of rotation angles considered for each misalignment. However, the data was obtained for independent rotations only. Data sets for all possible misalignments considering arbitrary rotations are prohibitively large and there are multiple solutions for any given combination of Zernike coefficients. As a complete Zernike coefficients data set had limited application and usefulness, it was concluded that a more deterministic approach to \ac{DVP} misalignment characterization was required.

\subsection{Autocollimator Diagnostics Investigation Results}
In Section~\ref{sec:autocollimatorDiagnostics} approaches to prism system alignment techniques were investigated. Though the D-275 autocollimator did not have the capability to completely define the alignment, it allowed for a more tangible investigation than did an interferometer. Testing with the autocollimator achieved partial alignment by observing reflected light from the prism surfaces and thereby established the reflection observation as a viable diagnostic tool for the partial alignment and characterization of a prism assembly.

\subsection{Linearized Approximation Results}
In Section~\ref{sec:linearPrism}, the output ray angle to a given input perturbation calculated using the linear approximation model was compared to Zemax simulation results of the same perturbed state. It was concluded that the linear approximation of the 25 mm H-ZF10 prisms with 25$^\circ$ wedge angles were accurate to within one arc second as long as the perturbation of each surface less than five arc minutes. A linear system model with a verified accuracy range allows for prism characterization utilizing the tools of linear analysis.

\subsection{System Identification}
In Section~\ref{sec:systemID}, identification of the two-prism system surface orientations using linear analysis of the perturbation model was investigated. The linear system model of Section~\ref{sec:linearPrism} was verified as accurate to within one arc second for all surface angular perturbation less than five arc minutes. From the perspective of system identification, this is equivalent to an accurate definition of angular orientation to within one arc second for a starting uncertainty of less than five arc minutes.

After performing system reduction by applying observations and assumptions as constraints to the two-prism system, the system of equations was undetermined with only a measurement of the output angle, Equation~\eqref{eq:reducedModel}. Adding to the system a measurement of the input angle resulted in the overdetermined model of Equation~\eqref{eq:overDefModel}. More investigation is required to complete a method for system identification. With the verified linear system, it is assumed that this is possible.

% \section{DVP Mechanical Design Conclusions}

\section{Tolerance Specification}

In Section~\ref{sec:linearPrism}, a linear approximation of a two-prism assembly was developed and validated. In Section~\ref{sec:systemID}, a reduced model for the linear approximation was presented in preparation for system identification. Based on these findings, it is assumed that the orientation of each surface of \ac{DVP} is able to be measured to within one arc second. With this supported assumption, the fabrication tolerances of the \ac{DVP} are not constrained by characterization capabilities. The tolerance specifications are only dependent upon the effect that deviations from the nominal design have on performance parameters. For example, a deviation of a surface angle may cause the dispersion angle to be less than the design, thereby reducing the spectral resolution of the system. The impact of deviations from the nominal design simulated in Zemax determine the acceptable limits. It is assumed that the acceptable limits imposed by performance parameters are within readily-achieved fabrication tolerances.

\section{Future Work} 

A new method of prism characterization has been proposed, but the method of system identification is not fully defined. Further investigation into the accuracy of the pseudoinverse applied to the reduced system model of  Equation~\eqref{eq:overDefModel} needs to be analyzed. New approaches and additional measurements may be proposed as necessary. Additionally the investigation should consider iterating with system identification to enhance the robustness of the application.

To make use of a method for system identification, it is necessary to enhance the linear model, Equation~\eqref{eq:overDefModel}, to incorporate real interferometric measurements. The exact method is determined by the measurement techniques. If the measurements of reflections through the prism are to be utilized, then the reflection perturbations from~\cite{RedBreck} must be used and the ray traced back through each surface.

The method of system characterization proposed allows for the tolerances of the Hawks \ac{DVP} to be specified. To do this, measures of performance must be specified for the Hawks \ac{DVP} which promote system requirements of \ac{CTEx}. Worst-case scenario performance is associated with fabrication tolerances by modeling misalignments of the four prisms in Zemax. These tolerances incorporated into drawings enable fabrication of a \ac{DVP}. If the tolerances are not reasonable for current industrial capabilities, then investigation into alignment and bonding techniques must be investigated.

